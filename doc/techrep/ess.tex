\documentclass{article}

%\usepackage{2up}

\usepackage[authoryear,round]{natbib}

%% Postscript fonts
%\usepackage{palatcm}
\usepackage{times}
\usepackage{amsmath}

%% The folowing is for creating PDF.

\newif\ifpdf
\ifx\pdfoutput\undefined
  \pdffalse     % not running PDFLaTeX
\else
  \pdfoutput=1  % running PDFLaTeX
  \pdftrue
\fi

\ifpdf
  \usepackage{thumbpdf}
\fi

%\usepackage{html,heqn,htmllist}
%\usepackage[latex2html]{hyperref} 


\ifpdf
  \usepackage[pdftex]{graphicx}
  \usepackage[pdftex]{hyperref} 
\else
  \usepackage[dvips]{graphicx}
  \usepackage[dvips]{hyperref} 
\fi

\hypersetup{backref,colorlinks=true,pagebackref=true,
  hyperindex=true}%pdfpagemode=FullScreen,

\newtheorem{defn}{Definition}[section]

\newtheorem{ex}{Example}[section]

\title{Emacs Speaks Statistics: A Universal Interface for
  Statistical Analysis}  

\author{A.J. Rossini\footnote{Department of Biostatistics, University
    of Washington and Department of Biostatistics, Fred Hutchinson
    Cancer Research Center, Seattle, WA, USA} \and Martin
  Maechler\footnote{Group  in Statistics; ETH Zurich; Zurich,
    Switzerland} \and Kurt 
  Hornik\footnote{Technical University of Vienna; Vienna, Austria} \and
  Richard M. Heiberger\footnote{Temple University; Philadelphia, PA,
    USA} \and Rodney Sparapani\footnote{Department of Statistics, Duke
    University; Durham, NC, USA, Medical College of Wisconsin, WI}}
\date{\today}

\begin{document}

\ifpdf
  \DeclareGraphicsExtensions{.jpg,.pdf,.png,.mps}
\fi

%%%% To cite everything
%%\nocite{*} 

\maketitle

Keywords: Statistical Analysis, Programming, User Interfaces

\begin{abstract}
  We discuss Emacs Speaks Statistics (ESS), a user interface for
  statistical programming based on Emacs, intended for many
  statistical programming languages.  It falls in the programming
  tools category of Integrated Development Environments (IDEs).  We
  discuss how it works, why one would consider using it, and
  extensions which increase the programming efficiency for statistical
  programming.
\end{abstract}

\section{Introduction}
\label{sec:intro}

%% Introduce the topic; I'm still trying to work out the best
%% approach, and this was my first thought.
Integrated Development Environments (IDEs) are tools for increasing
programmer efficiency.  The increased speed of software development
can be partially attributed to the use of such Rapid Application
Development (RAD) tools.  Statistical programming, with the increased
use of computers and the complexity of data is a skill which can be
augmented by the right tools and environment.  However, one issue that
arises is that different tools have different strengths, and optimal
use can require switching back and forth between tools for data
analysis.

%% Goals for the project as they've been stated (and have morphed). 
The initial goal for Emacs Speaks Statistics (ESS) was to provide a
single environment for using both multiple instances of a statistics
program as well as multiple instances of different statistics
programs.  For example, one might want to be connected to multiple R
\citep{ihak:gent:1996} processes.  Reasons for this include
verifying behavior on different versions of the same software, test
and run scenarios where one process is doing long-term processing
while the other is doing short-term testing, simulations, running
multiple processes on multiple machines from the same place.  There
are many additional other applications, as well.

%% Rationale for the goals?  This is weak.
For small projects, it can be useful to have a single repository (file
or directory with a few files) for documenting code which produces
datasets and analysis results.  This is facilitated by being able to
edit files (different or the same) and being able to specify different
modes for handling programming code for the different statistical
languages.


\subsection{Emacs}
\label{sec:intro:emacs}

%% Introduce and describe Emacs.  2-3 paragraphs?
Emacs is powerful and extensible text editor which is
available for a large number of platforms.  In addition, most
programming and documentation falls under the realm of text
processing, with additional features for IDEs and word processing
needs such as contextual highlighting and recognition of data points.

Extensions to Emacs allow it to act as a World-Wide-Web browser, a
highly sophisticated mail and news reader, a shell/terminal window
with history, and as many other common text-based tools.  It can be
re-mapped to act as many other text-editors, such as ed, vi, wordstar,
and brief.  In addition, the above plus its ability to interface with
other processes makes it an ideal platform for providing a universal
interface to statistical packages.  The use of emacs-lisp as an
extension language has also facilitated the programming.

Because of the above, the choice of Emacs as the basis for a universal
interface is a natural one.  

\textbf{Need an Introduction to EMACS}

We assume that you are familiar with Emacs terminology and syntax:
file, buffer, region, description of keys etc.  If not, please read
the New Users guide (found in the info pages, "C-h i" (by pressing
control h, i) or Tutorial, "C-h t").

To find the key-sequences for commands, view the keymap (C-h b) or
view help for the current mode (C-h m).



\subsection{Other Statistical User Interfaces}
\label{sec:intro:UI}

%% Statistical User Interfaces.
For the purposes of the user interface, recent and traditional
interfaces for statistical packages and languages can generally be
classified into 3 forms.  There is the command-line interface that
most of the packages have available; this is the interface that ESS
needs for interfacing at the process level.  There is also the
spreadsheet/MDI interface employed by both spreadsheet packages as
well as most Apple and Microsoft-based statistical packages.  In
addition, there have been other one-time implementation interfaces,
which include graph-based interfaces as exhibited by ViSta
\citep{youn:lubi:1995}, the SAS terminal interface, which divides the
terminal window into 3 screens, and possibly others.


\subsection{ESS}
\label{sec:intro:ESS}


%% History of ESS.
ESS grew out of the programming and process interface of the Emacs
S-mode (D. Bates, et.al; D. Smith).  The extension to a
language-independent generic interface was prompted by the success of
R, and the need for an R-mode.  This led to a merger with the SAS-mode
(T. Cook), and the refactoring of the S-mode codebase to accommodate
multiple languages in a flexible way.  A detailed  history can be
found in section~\ref{sec:history}.

%% Current State
ESS currently supports a number of statistical languages, with various
levels of support depending on their capabilities and needs.  Because
of it's history, ESS supports the S family of languages extremely
well; these include recent versions of S, S-PLUS, and R.  SAS is also
well supported, but to a lesser extent.  The lack of objects in SAS
has prevented the use of object completion facilities.  Stata and
XLispStat (and the XLispStat extensions, ARC and ViSta) are marginally
supported, providing mainly syntax highlighting and
process-interfacing.  Note also that marginal support is the
functionality that the majority of ESS users take advantage of.

%% Roadmap for the paper.
We follow this up by describing how ESS can be used at first; then
describe advanced usage.  The next sections explain where it can be
obtained as well as some history of the project.  Finally, we close
with a discussion of where it stands and how IDEs might be realized in
the future.

\section{Introductory Usage}
\label{sec:basic}


%%% MAKE THIS INTO A PARAGRAPH
ESS provides a number of features, including:
\begin{itemize}
\item syntax highlighting 
\item standard formating and customizable automatic indentation
\item ability to send text regions and buffers to statistical programs
\item command-line history searching 
\end{itemize}

%%% MAKE THIS INTO A PARAGRAPH
In addition, for S, S-PLUS, and R, ESS provides
\begin{itemize}
\item object-name completion
\item ability dump, edit, save, and reload functions and similar
  programming code.
\end{itemize}

Along with this, Emacs provides an interface to revision control
systems for maintaining version history and a changelog for files.

%%% MAKE THIS INTO A PARAGRAPH
Editing source code (S, LispStat, SAS)
\begin{itemize}
\item Syntactic indentation and highlighting of source code
\item Partial evaluations of code
\item Loading and error-checking of code
\item Source code revision maintenance
\end{itemize}

%%% MAKE THIS INTO A PARAGRAPH
Interacting with the process (S, LispStat, SAS)
\begin{itemize}
\item Command-line editing
\item Searchable Command history
\item Command-line completion of S object names and file names
\item Quick access to object lists and search lists
\item Transcript recording
\item Interface to the help system
\end{itemize}

%%% MAKE THIS INTO A PARAGRAPH
Transcript manipulation (S3, S+3, S4, R, XLispStat)
\begin{itemize}
\item Recording and saving transcript files
\item Manipulating and editing saved transcripts
\item Re-evaluating commands from transcript files
\end{itemize}

%%% MAKE THIS INTO A PARAGRAPH
Help File Editing (R)
\begin{itemize}
\item Syntactic indentation and highlighting of source code.
\item Sending Examples to running ESS process.
\item Previewing
\end{itemize}



\subsection{Editing Files}
\label{sec:basic:edit}


ESS[S] is the mode for editing S language files.  This mode handles:

- proper indenting, generated by both [Tab] and [Return].
- color and font choices based on syntax.
- ability to send the contents of an entire buffer, a highlighted
  region, an S function, or a single line to an inferior S process, if
  one is currently running.
- ability to switch between processes which would be the target of the 
  buffer (for the above).
- The ability to request help from an S process for variables and
  functions, and to have the results sent into a separate buffer.
- completion of object names and file names.

ESS[S] mode should be automatically turned on when loading a file with
the suffices found in ess-site (*.R, *.S, *.s, etc).  However, one
will have to start up an inferior process to take advantage of the
interactive features.


\subsection{Inferior ESS processes}
\label{sec:basic:inf}

iESS (inferior ESS) is the mode for interfacing with active
statistical processes (programs).  This mode handles:

- proper indenting, generated by both [Tab] and [Return].
- color and font highlighting based on syntax.
- ability to resubmit the contents of a multi-line command
  to the executing process with a single keystroke [RET].
- The ability to request help from the current process for variables
  and functions, and to have the results sent into a separate buffer.
- completion of object names and file names.
- interactive history mechanism
- transcript recording and editing

To start up iESS mode, use:
   M-x S+3 
   M-x S4
   M-x R

(for S-PLUS 3.x, S4, and R, respectively.  This assumes that you have
access to each).  Usually the site will have defined one of these programs
(by default S+3) to the simpler name:

   M-x S

Note that R has some extremely useful command line arguments, 
-v and -n.   To enter these, call R using a "prefix argument", by

   C-u M-x R

and when ESS prompts for "Starting Args ? ", enter (for example):

   -v 10000 -n 5000

Then that R process will be started up using "R -v 10000 -n 5000".

New for ESS 5.1.2 (and later):  "S-elsewhere" command

  The idea of "M-x S-elsewhere" is that we open a telnet (or rlogin)
  to another machine, call the buffer "*S-elsewhere*", and then run S
  on the other machine in that buffer.  We do that by defining "sh" as
  the inferior-S-elsewhere-program-name.  Emacs sets it up in a
  "*S-elsewhere*" iESS buffer.  The user does a telnet or login from
  that buffer to the other machine and then starts S on the other
  machine.  The usual C-c C-n commands from myfile.s on the local
  machine get sent through the buffer "*S-elsewhere*" to be executed
  by S on the other machine.
                           

\subsection{Handling and Reusing Transcripts}
\label{sec:basic:trans}

- edit transcript
- color and font highlighting based on syntax.
- resubmit multi-line commands to an active process buffer
- The ability to request help from an S process for variables and
  functions, and to have the results sent into a separate buffer.
- ability to switch between processes which would be the target of the 
  buffer (for the above).


\subsection{Programming Language Help}
\label{sec:basic:help}


- move between help sections
- send examples to S for evaluation


\subsection{Philosophies for using ESS}
\label{sec:basic:philosophy}


The first is preferred, and configured for.  The second one can be
retrieved again, by changing emacs variables.

1: (preferred by the current group of developers):  The source code is 
   real.  The objects are realizations of the source code.  Source
   for EVERY user modified object is placed in a particular directory
   or directories, for later editing and retrieval.

2: (older version): S objects are real.  Source code is a temporary
   realization of the objects.  Dumped buffers should not be saved.
   We strongly discourage this approach.  However, if you insist,
   add the following lines to your .emacs file:

      (setq ess-keep-dump-files 'nil)
      (setq ess-delete-dump-files t)
      (setq ess-mode-silently-save nil)

The second saves a small amount of disk space.  The first allows for
better portability as well as external version control for code.


\subsection{Scenarios for use}
\label{sec:basic:scenarios}

We present some basic suggestions for using ESS to interact with S.
These are just a subset of approaches, many better approaches are
possible.  Contributions of examples of how you work with ESS are
appreciated (especially since it helps us determine priorities on
future enhancements)! (comments as to what should be happening are
prefixed by "\verb+##+").

\begin{itemize}
\item Data Analysis Example (source code is real)
\begin{verbatim}
    ## Load the file you want to work with
    C-x C-f myfile.s

    ## Edit as appropriate, and then start up S-PLUS 3.x
    M-x S+3

    ## A new buffer *S+3:1* will appear.  Splus will have been started
    ## in this buffer.  The buffer is in iESS [S+3:1] mode.

    ## Split the screen and go back to the file editing buffer.
    C-x 2 C-x b myfile.s

    ## Send regions, lines, or the entire file contents to S-PLUS.  For regions,
    ## highlight a region with keystrokes or mouse and then send with:
    C-c C-r

    ## Re-edit myfile.s as necessary to correct any difficulties.  Add
    ## new commands here.  Send them to S by region with C-c C-r, or
    ## one line at a time with C-c C-n.

    ## Save the revised myfile.s with C-x C-s.

    ## Save the entire *S+3:1* interaction buffer with C-c C-s.  You
    ## will be prompted for a file name.  The recommended name is
    ## myfile.St.  With the *.St suffix, the file will come up in ESS
    ## Transcript mode the next time it is accessed from Emacs.
\end{verbatim}
\item Program revision example (source code is real)
\begin{verbatim}
    ## Start up S-PLUS 3.x in a process buffer (this will be *S+3:1*) 
    M-x S+3

    ## Load the file you want to work with
    C-x C-f myfile.s
    
    ## edit program, functions, and code in myfile.s, and send revised
    ## functions to S when ready with
    C-c C-f
    ## or highlighted regions with
    C-c C-r
    ## or individual lines with
    C-c C-n
    ## or load the entire buffer with 
    C-c C-l

    ## save the revised myfile.s when you have finished
    C-c C-s
\end{verbatim}
\item Program revision example (S object is real)
\begin{verbatim}
    ## Start up S-PLUS 3.x in a process buffer (this will be *S+3:1*) 
    M-x S+3

    ## Dump an existing S object my.function into a buffer to work with
    C-c C-d my.function
    ## a new buffer named yourloginname.my.function.S will be created with
    ## an editable copy of the object.  The buffer is associated with the
    ## pathname /tmp/yourloginname.my.function.S and will amlost certainly not
    ## exist after you log off.

    ## enter program, functions, and code into work buffer, and send
    ## entire contents to S-PLUS when ready
    C-c C-b

    ## Go to *S+3:1* buffer, which is the process buffer, and examine
    ## the results.
    C-c C-y
    ## The sequence C-c C-y is a shortcut for:  C-x b *S+3:1*

    ## Return to the work buffer (may/may not be prefixed)
    C-x C-b yourloginname.my.function.S
    ## Fix the function that didn't work, and resubmit by placing the
    ## cursor somewhere in the function and
    C-c C-f
    ## Or you could've selected a region (using the mouse, or keyboard 
    ## via setting point/mark) and 
    C-c C-r
    ## Or you could step through, line by line, using 
    C-c C-n
    ## Or just send a single line (without moving to the next) using
    C-c C-j
    ## To fix that error in syntax for the "rchisq" command, get help
    ## by
    C-c C-v rchisq
\end{verbatim}
\item Data Analysis (S object is real)
\begin{verbatim}
    ## Start up S-PLUS 3.x, in a process buffer (this will be *S+3:1*) 
    M-x S+3

    ## Work in the process buffer.  When you find an object that needs 
    ## to be changed (this could be a data frame, or a variable, or a 
    ## function), dump it to a buffer:
    C-c C-d my.cool.function

    ## Edit the function as appropriate, and dump back in to the
    ## process buffer  
    C-c C-b

    ## Return to the S-PLUS process buffer
    C-c C-y
    ## Continue working.

    ## When you need help, use 
    C-c C-v rchisq
    ## instead of entering:   help("rchisq")
\end{verbatim}
\end{itemize}


\section{Advanced Usage}
\label{sec:advanced}

\section{Obtaining ESS}
\label{sec:getIt}

ESS is primarily located at ess.stat.wisc.edu, and is available by FTP
or through the World-Wide-Web (WWW).  A basic installation consists of
downloading the package, unpacking, and then adding a line to the
emacs initialization file (\verb+.emacs+ or \verb+_emacs+), pointing
to the lisp subdirectory of the unpacked archive.

\section{History}
\label{sec:history}

ESS (originally S-mode) was initially designed for use with S and
S-PLUS(tm); hence, this family of statistical languages currently has
the most support.  We denote by S, any of the currently available
members of the family, including S 3.x, S 4.x, S-PLUS 3.x, S-PLUS 4.x,
S-PLUS 5.x, and R.  In addition, we denote by Emacs, one of the GNU
family of editors, either Emacs (as developed and maintained by the
Free Software Foundation) or XEmacs (which is a derivative work).

ESS is a remarkable example of how open-source products can continue
to develop beyond what their initial authors planned.  ESS started as
S-mode, an Emacs extension which was developed in 1991 as a simple
tool for editing program files for S and S-PLUS.  In 1994, Rossini
extended S-mode to support XEmacs, which was a forked version of
Emacs.  At the same time, Rossini extended a successful SAS-mode
written by Tom Cook to work with XEmacs.  In 1995, S-mode was merged
into a uniform S-mode for Emacs, XEmacs, and supported S, S-PLUS, and
R.  During 1996 and 1997, this was extended to incorporate the
SAS-mode as well as to provide a generic means for configuring ESS to
accommodate changed as well as new statistical languages.

Most of this was primarily done under Unix.  However, in 1998, thanks
to an example of interprocess communication using Microsoft's DDE by
Brian Ripley, Richard M. Heiberger provided interfaces for S-PLUS 4.x
and S-PLUS 2000.

Most of what has been done recently is debugging and tuning.  New
features in the plans include robust extensions for Literate Data
Analysis using Noweb (Ramsey, 199x?), as well as extensions for
XML-based Literate Statistical Analysis.

\section{Remarks and Extensions}
\label{sec:remarks}

There are two active areas of extensions for user environments.  One
is to enhance the capabilities of the IDE for statistical practice;
this includes implementing such common IDE features as object browsers
as well as clean up the interface.

The other exciting extension is towards quasi-Literate Programming
methodologies (Knuth, 1987?; Ramsey, 199x?), which we will refer to as
Literate Statistical Analysis \citep{ross:lunt:2001}. The tools include
the use of noweb (Ramsey, 199x?) and an additional step towards the
use of an XML authoring environment for statistical analysis.
Literate Data Analysis based on noweb, is a means of documenting a
statistical analysis plan and procedure.  Literate Statistical
Analysis is intended to be a round-trip environment for both design
and analysis.

The future use of XML is to accommodate the growing use of WWW-based
services for document publishing as well as information conversion.
By marking up the document with XML, it is possible to display subsets
of the document contextually according to the intent of the document.
Unlike Literate Data Analysis, which provides a single document for
code and analysis plan, Literate Statistical Analysis is a round-trip
cycle, using document modification through the Document Object Model
interface (W3 Consortium).  This is described more in
\citep{ross:lunt:2001}. 


\bibliographystyle{plainnat} %alpha}
\bibliography{ess}

\end{document}

%%% Local Variables: 
%%% mode: latex
%%% TeX-master: "ess"
%%% End: 

