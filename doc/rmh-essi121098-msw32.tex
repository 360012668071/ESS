\documentclass[12pt]{article}         %LaTeX 2e

\newcommand{\tty}[1]{{\tt #1}}

\begin{document}
December 10, 1998

I ahve now fully incorporated the ideas I did quickly in  essi-w32 into the ESS
package.

While I was at it, I incorporated the ideas in \tty{M-x S-elsewhere}, essentially the
proposal I sent to ess-bugs in April 1998.

We can now do
\begin{description}
\item[\tty{M-x S+4}:]
Run S+4 in it's own GUI window from an emacs buffer \tty{myfile.s}.
It does not open an \tty{*S+4*} buffer.
\item[\tty{M-x Sqpe+4}:]
Run Sqpe+4 in an emacs buffer \tty{*S+4*} from an emacs buffer \tty{myfile.s}.
It behaves exactly like S+3 on unix, except of course, for no interactive graphics.
\item[\tty{M-x ess-external-minor-mode}:]
Toggle between the above from the same \tty{myfile.s} buffer.
\item[\tty{M-x S+elsewhere}:]
Run an S process on another computer in an\\
\tty{*S-elsewhere*} buffer
from \tty{myfile.s} on my computer.  This will work not only from a PC
to a unix machine, but also from one unix machine to another.
\end{description}

My goal is a release of the next version of ESS in early January
1999 to catch the beginning of the Spring Semester.


Here is what is going on:
\begin{enumerate}
\item
I am using the Andrew Innes' ddeclient distributed with NTemacs
20.3.1.  The fundamental trick is to define a new keymap,
ess-external-mode-map, which is almost the same as ess-mode-map.  The
difference is that \tty{C-c C-n} and \tty{C-c C-r}, originally bound
to \tty{ess-eval-region} and \tty{ess-eval-line-and-next-line},\\
are now bound to \tty{ess-eval-region-ddeclient} and\\
\tty{ess-eval-line-and-next-line-ddeclient}.

\item
Maybe an improved ddeclient should be written, to return S's response
to whatever we type.  I assume that the logic of
Brian Ripley's Sdde, together with the details of Andrew Innes'
ddeclient, can be written and that it will work.
This is not necessary.  I am willing to pick up useful output from the S-Plus commands
window and drop it into an emacs \tty{myfile.st} buffer manually.

If the improved dddeclient is written,
it should have the S-Plus Commands window closed.


\item
The problem with the earlier attempts at \tty{M-x Sqpe+4} (I made it
work last summer, as did Brian, and maybe others of you), was the lack
of prompt.  I solved that by a two-step process.  First start the
process with \tty{M-x Sqpe+4}.  Then when ESS times out for lack of a
prompt, run \tty{M-x Sqpe+4b}.  The magic in \tty{M-x Sqpe+4b} is
twofold: I make S interactive and I tell emacs that S is not echoing
the commands.  Putting these two routines together requires changing
the inferior-ess-wait-for-prompt() loop in ess-inf.el.  This is buried deep
in ess-multi().

\item
Unlike in unix, I think the S+4 GUI should be launched to survive
killing the emacs process.  This means (assuming bash) that it needs
to be launched with the command \tty{Splus \&} and not just with
\tty{Splus}.  I ducked the question and \tty{M-x S+4} gives a minibuffer
message that says the user should click on the S-Plus icon.

\item
Help with \tty{C-c C-v} needs sends the
\tty{?command} to S.  The Splus libraries cannot have their help
redirected.  Only user-written libraries can have help redirected.
An empty *help()* buffer currently appears.

\item
The idea of \tty{M-x S-elsewhere} is that we open a telnet (or rlogin) to
another machine, call the buffer \tty{*S-elsewhere*}, and
then run S on the other machine in that buffer.  I do that by defining \tty{sh}
as the inferior-S-elsewhere-program-name.  Emacs sets it up in a \tty{*S-elsewhere*}
iESS buffer.  The user does a telnet or login from that buffer to the other machine
and then starts S on the other machine.

\item
I think this generalizes to any program that ESS can run.  All the magic is isolated in
the essd-iw32.el file.

\item
I did this with 6 files based on ESS 5.0.
\begin{verbatim}
  4574 Dec 10 00:50 essd-sq4.el
 14943 Dec 10 00:09 ess-site.el
  4050 Dec 10 00:02 ess-iw32.el
 31159 Dec  9 23:28 ess-vars.el
  4028 Dec  9 23:20 essd-els.el
  4158 Dec  9 23:17 essd-s+4.el
\end{verbatim}
\end{enumerate}
\end{document}
