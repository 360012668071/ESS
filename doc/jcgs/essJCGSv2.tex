\documentclass{article}

%1.  a statistician's dream environment
%2.  what emacs has to offer
%        a) buffers
%        b) key-binding
%        c) other packages
%                1) font-lock
%                2) comint
%                3) ange-ftp/EFS/Tramp
%3. what ESS has to offer
%        1) font-lock/key-binding/other packages revisited
%        2) statistical process interaction
%                a) interactive
%                        1) R/S
%                        2) SAS
%                b) batch
%                        1) SAS
%                        2) BUGS
%4. ESS as an open source project
%        1) early days
%                a)S-mode
%                b)SAS-mode
%        2) transition
%                a)ESS-mode
%                b)Emacs/XEmacs
%        3) modern era
%                a)GPL, cvs, ssh
%                 b)ESS 6, the next generation?


%\renewcommand{\baselinestretch}{1.5}
\addtolength{\oddsidemargin}{-0.5in}
%\addtolength{\topmargin}{-0.2in}
\addtolength{\textheight}{0.2in}
\addtolength{\textwidth}{1in}

%%%
\usepackage[authoryear,round]{natbib}
%or (if you have an unshiny latex installation)
%\newcommand{\citep}[1]{{\{\sf#1\}}}
%%%
\usepackage{alltt}

%% Postscript fonts
%\usepackage{times}
\usepackage{graphicx}

\ifx\pdfoutput\undefined
  %% Stuff wout hyperref
  \def\url#1{\textsf{#1}} % To help fit in lines
\else
  %% Stuff with hyperref
  \usepackage{hyperref}
  %%\hypersetup{backref,colorlinks=true,pagebackref=true,hyperindex=true}
  \hypersetup{backref,colorlinks=false,pagebackref=true,hyperindex=true}
\fi

%%---End of package requiring ---------- Own Definitions -------------

\newcommand*{\SAS}{\textsc{SAS}{\textregistered} }
\newcommand*{\Splus}{\textsc{S-Plus}}
\newcommand*{\XLispStat}{\textsc{XLispStat}}
\newcommand*{\Stata}{\textsc{Stata}}
\newcommand*{\Rgui}{\textsc{Rgui}}
\newcommand*{\Perl}{\textsc{Perl}}
\newcommand*{\Fortran}{\textsc{Fortran}}
\newcommand*{\Scmt}[1]{\hbox{\qquad {\footnotesize \#\#} \textsl{#1}}}
\newtheorem{defn}{Definition}[section]
\newtheorem{ex}{Example}[section]

\newcommand{\stexttt}[1]{{\small\texttt{#1}}}
\newcommand{\ssf}[1]{{\small\sf{#1}}}
\newcommand{\elcode}[1]{\\{\stexttt{\hspace*{2em} #1}}\\}
\newenvironment{Salltt}{\small\begin{alltt}}{\end{alltt}}
\newcommand{\US}{{\char'137}}        % \tt _
\newcommand{\marpar}[1]{\marginpar{\raggedright#1}}
\newcommand{\file}[1]{`\stexttt{#1}'}

%%--------------------------------------------------------------- Start Text

\title{Emacs Speaks Statistics:  A Rapid Application Development (RAD)
  Environment for Statistical Analysis Packages}

\author{A.J. Rossini \and Martin M{\"a}chler \and Kurt Hornik \and Richard
  M. Heiberger \and Rodney Sparapani \footnote{%
%%
    A.J. Rossini is Research Assistant Professor in the Department of
    Biostatistics, University of Washington and Joint Assistant Member at
    the Fred Hutchinson Cancer Research Center, Seattle, WA, USA
    (E-mail: rossini@u.washington.edu);
%%
    Martin M{\"a}chler is Senior Scientist and Lecturer in the Seminar for
    Statistics, ETH Zurich, Zurich, Switzerland
    (E-mail: maechler@stat.math.ethz.ch);
%%
    Kurt Hornik is Professor in the Institut f{\"u}r Statistik,
    Wirtschaftsuniversit{\"a}t Wien and the Institut f{\"u}r
    Wahrscheinlichkeitstheorie und Statistik, Technische Universit{\"a}t
    Wien, Vienna, Austria (E-mail: Kurt.Hornik@r-project.org);
%%
    Richard M. Heiberger is Professor in the Department of Statistics at
    Temple University, Philadelphia, PA, USA (E-mail: rmh@temple.edu);
%%
    and Rodney Sparapani is Senior Biostatistician at the Medical College
    of Wisconsin, Milwaukee, WI, USA (E-mail: rsparapa@mcw.edu)}}

\date{\today}

\begin{document}

\maketitle

\begin{abstract}
  Emacs Speaks Statistics (ESS) provides a user interface for
  developing statistical applications and performing data analysis
  using any of several common statistical programming languages.  ESS
  is based on the Emacs text editing environment and is a member of
  the programming tools category of Integrated Development
  Environments (IDEs), which are tools which aid in developing and
  visualizing computer applications.  We discuss how it works, why one
  would consider using it, and additional extensions for increasing
  statistical programming efficiency.
\end{abstract}

\noindent Keywords: Data Analysis, Programming Tools, User Interfaces, \SAS,
\Splus, R, \XLispStat, \Stata

\baselineskip=2pc

\section{Introduction}
\label{sec:introduction}

The common tasks faced by a statistician are not different from those
faced by software developers, since a great deal of statistical
activity involves some form of computer programming.  Common
statistical research activities, such as data analysis and
communication, are enhanced by computers.  More importantly, the user
interface, which acts as a buffer between the statistician and
computer, plays a critical part in facilitating or hindering this
interaction.  This paper introduces Emacs Speaks Statistics (ESS),
which is a set of extensions to the Emacs text editor which facilitate
the use of a number of common statistical programs and languages by
provides shortcuts and extensions for often-performed tasks.

%% Does it make sense to add an additional paragraph on "history of
%% statistical interfaces from programming it yourself to command line
%% (SPSS/SAS/BDMP in the 80s), to spreadsheet-style interfaces
%% (spreadsheets, AND minitab, spss now, etc), etc?  This would be the
%% 2nd paragraph, if so. 

ESS provides an interface to statistical packages, and provides,
through Emacs, additional tools which facilitate both statistical
software development and data analysis.  Currently supported
statistical packages are the following: the S languages (which include
S \citep{BecRCW88,ChaJH92,ChaJ98}, \Splus{} \citep{Splus}, and R
\citep{ihak:gent:1996}); \SAS \citep{SAS:8.0}; \XLispStat\ 
\citep{Tier90} and its extensions Arc \citep{Cook:Weisberg:1999} and
ViSta \citep{youn:fald:mcfa:1992}; \Stata\ \citep{Stata:6.0}; Omegahat
\citep{DTLang:2000}; and BUGS \citep{SpieThomBest:1999}.  ESS can be
extended to accommodate most statistical packages which provide either
an interactive command-line or batch file processing.

The present paper provides historical and non-technical information on
Emacs and ESS.  A general introduction and usage instructions can be
found in \cite{heiberger:dsc:2001}.  The documentation that comes with
ESS provides details of its implementation and examples of its use.
We first describe the Emacs editor, the platform on which ESS is
built.  Emacs is a powerful and extensible text editing tool.  This
section is followed by the details of ESS and how it facilitates
common statistical computing activities.  We then describe the unique
history that ESS has had as an open source project, as it has been
passed down through numerous development groups and has taken into its
fold and extended a number of external Emacs products which support
statistical languages and programming.  Finally, we conclude with
future tasks and extensions.

\section{Emacs}
\label{sec:emacs}

Emacs is a mature, powerful, and extensible text editing system which
is freely available under the GNU General Public License for a large
number of platforms, including most flavors of Unix, Microsoft Windows
and Apple Macintosh.  There are two versions, or forks, of Emacs (GNU
Emacs \citep{RMS:2000} and XEmacs).  Emacs shares many features with
word processors, and some characteristics with operating systems.
Most importantly, Emacs can interact with and control other programs.

Emacs was originally written when single-screen terminals were the
most sophisticated method of access to computers.  Common Emacs
functions are assigned to key sequences, which is known as key-mapping
or key-binding, though all functions can be executed by name.  For
most current users, Emacs is rarely used on in a terminal-emulator
except in special circumstances such as small, quick edits or remote
access to the computer over a slow connection.  Over the last decade,
Emacs has been extended to use windowing systems such as X11,
Microsoft Windows, and Apple Macintosh.  These windowing systems
provide a graphical user interface (GUI) which allows the user to
interact with either a keyboard or a mouse and provides applications
with the ability to share space on a single display.  Emacs menus and
toolbars allow mouse access to frequently used actions; for instance,
when you don't know or can't remember the appropriate key-mapping.
User-defined menus and toolbars can be constructed as needed.

Emacs provides facilities which go beyond ordinary text editing.  It
creates a dedicated space on the display, called a buffer, to perform
tasks such as editing a text file on disk or controlling other
programs.  One can switch between buffers to observe or progress on
concurrent tasks.  Buffers allow the user to simultaneously edit or
control multiple files and programs.

Emacs capabilities are extended by text files containing Emacs Lisp,
(also known as E-Lisp), a dialect of Lisp
\citep{RChassell1999,PGraham:1996}.  Emacs commands can be written
in E-Lisp to be bound to key-maps or called interactively by the user.
Emacs functions are also written in E-Lisp, but they can only be 
called by Emacs commands or other Emacs functions rather than directly
by key-maps or interactively.  The most important extensions to
Emacs are called modes.  Major-modes provide two types of support.
First, it customizes the environment (key-maps, capabilities) for types
of files determined by the file name's extension, i.e. the characters
at the end of the file name that follow a period like ``txt'', ``s''
or ``sas''.  When they are dictated by the file name, you can only
utilize one major-mode at a time.  Minor-modes, by contrast, provide
services that are not mutually exclusive.  This includes emulation of
another editor or cooperating with a version control program.  For
example, the font-lock minor-mode allows Emacs to highlight, with
fonts or colors, the syntax of a programming language whose
characteristics are described by a major-mode like ESS.  Major-modes
provide automatic indentation and navigation in units of characters,
words, lines, sentences, paragraphs, and pages as well as many other
features.

The second type of Major-mode is for running Lisp programs.  In this
case, the buffer provides input and output to the Lisp program.  These
can control other processes or interact with the system to send mail
or transfer files over networks, play games, configure the running
Emacs session, or perform many other tasks.

Most programming and documentation tasks fall under the realm of text
editing.  This work can be enhanced by features such as contextual
highlighting and recognition of special reserved words appropriate to
the programming language in use.  In addition, editor behaviors such
as folding, outlining, and bookmarks can assist with maneuvering
around a file.  Emacs shares many features with word-processing
programs and cooperates with document preparation systems such as
\LaTeX, HTML, \textsc{xml}, XSLT, and Noweb.
%Type-setting and word-processing, which
%focus on the presentation of a document, are tasks that are not pure
%text-editing.

%insertion and deletion: viewing two or more files at once; editing formatted text;
%visual comparison of two similar files; and navigation 
%in units of characters, words, lines, sentences, paragraphs, and pages.  

The capabilities can be extended to include other Emacs modes for
interacting with version control packages like RCS, CVS, SCCS, or
PRCS; and remote editing via ftp/telnet or scp/ssh.  Emacs handles the
interface to both source code and transcripts contextually, providing
syntax highlighting, bookmarking features, interfaces to directory
structure, and command-history.  Other extensions to Emacs allow it to
act as a World-Wide-Web browser, a highly sophisticated mail and news
reader, a shell/terminal window with history, and as an interface to
other common text-based tools such as spell checking programs.  The
Emacs keyboard and mouse interface can be re-mapped to resemble that
of other text-editors, such as vi, wordstar, and brief.


\section{ESS extends Emacs}
\label{sec:ess-extends-emacs}

Emacs' power and flexibility make it a sensible starting choice to
build an interface for statistical analysis packages.  ESS extends
Emacs to provide a functional, easily extensible and uniform interface
for multiple statistical packages.  This is done in two ways.  First,
shortcuts and features for accelerating the editing files have been
written and implemented.  Second, interaction with the statistical
language is split by function; general, obtaining help, programming.
\textbf{THIS NEEDS WORK}.

%This is done by extending the editor to provide
%additional useful features, and in the case of interactive statistical
%programs, sitting ``in front of'' their CLI and intercepting and
%modifying I/O as needed.  We describe the exact features more in this
%section.  

%% change/edit
%Figure~\ref{fig:1} provides an example of how it looks when being used
%with XEmacs.  This screen-shot shows both SAS and R code being examined
%at the same time, with an R interactive process being controlled from
%within XEmacs.

%Figure~\ref{fig:2} shows ESS running in NTemacs 21.0 on Microsoft
%Windows 2000.  We are currently displaying 6 buffers.
%\stexttt{highlight.s} illustrates several uses of syntactic
%highlighting.  The most glaring one is the bright purple indicator for
%the unbalanced parentheses.  We also see color choices for keywords
%(\stexttt{if}), comments, and quoted strings.  The string on the last
%line was not properly terminated and we are immediately warned by the
%string color staying on through (what we think is) the end of the
%line.

%The two buffers \stexttt{transcript-before.st} and
%\stexttt{transcript-after.st} show transcript editing.  A single ESS
%command converted the before buffer into the after buffer by removing
%all lines that do not begin with a prompt character.

%The last three buffers show that a single S language source file can
%be used with two (or more) executing processes.  In this example we
%sent over first a subset of the line in \stexttt{tmps.s} and then the
%entire line to the instance of S+4 running in an inferior
%\stexttt{iESS(Sqpe)} buffer.  Then we switched the connection to send
%the same line to the instance of R running in the \stexttt{iESS(R)}
%buffer.  The dialog about the process switch appears in the message
%line.


\subsection{Features and capabilities}
\label{sec:ESS:features}

Since ESS originated as S-mode, which provided an interface for
programming and process control under GNU Emacs for \Splus\ version 3,
ESS strongly supports the S family of languages; these include recent
versions of S, \Splus, and R.  \SAS is also well supported.  \Stata\ 
and \XLispStat\ (and the \XLispStat\ extensions, ARC and ViSta) are
supported with the basic functionality of syntax highlighting and
process-interfacing.  ESS and its interface are fully discussed in the
next section.

\paragraph{Syntactic indentation and color/font-based source code
  highlighting.}  The ESS interface includes a description of the
syntax and grammar of each statistical language it knows about.  This
gives ESS the ability to edit the programming language code, often
more smoothly than with editors distributed with the languages.  The
process of programming code is enhanced as ESS provides the user with
a clear presentation of the code with syntax highlighting to denote
assignment, reserved words, presence of strings, and comments.  ESS
has customizable automatic indentation, with the customization based
on the syntactic structure of groups of expressions.  ESS knows the
structure of transcripts of the executing session and provides
syntactic highlighting for transcripts.  ESS interacts well with the
executing statistics language/program and provides means for searching
the command-line history for previous commands and editing them for
current use.

\paragraph{Partial code evaluation.}
Emacs can send individual lines, entire function definitions, marked
regions, and whole edited buffers from the window in which the code is
displayed for editing to the statistical language/program for
execution.  Emacs can complete partially typed file names by referring
to the current working directory.  Emacs sends the code directly to
the running program and receives the printed output back from the
program.  This is a major improvement over cut-and-paste as it does
not require switching buffers or windows.  The response is received
immediately in an editable Emacs buffer.

\paragraph{Object name completion.}
In addition, for languages in the S family (S developed at Bell Labs,
\Splus, and R) ESS provides object-name completion of both user- and
system-defined functions and data.  ESS can dump and save objects
(user- and system-generated) into text files in a formatted manner for
editing, and reload them (possibly after editing) back into the
statistical language/program.

\paragraph{Source code checking.}
ESS facilitates the editing of source code by providing a means for
loading and error-checking of small sections of code for S,
\XLispStat, and \SAS.  This allows for source-level debugging of batch
files.

\paragraph{Process interaction.}
Emacs has historically referred to processes under its control as
``inferior'', accounting for the name inferior ESS (\stexttt{iESS}) to
denote the mode for interfacing with the statistical package.  The
output of the package goes directly to an editable text buffer in Emacs.
This mode allows for command-line editing and saving history, as well as
recalling and searching for previously entered commands.  Filename
completion is available.  In addition (currently only for S languages),
there exists object-name and function-name completion.  Transcripts are
easily recorded and can be edited into an ideal activity log which can
then be saved.  There is a good interface for handling and intercepting
calls to the internal help systems for S, \XLispStat, and \Stata.

\paragraph{Interacting with statistical programs on remote computers.}
ESS provides the facility to edit and run programs on remote machines
in the same session and with the same simplicity as if they were
running on the local machine.  The remote machine could be a very
different platform than the local machine.

\paragraph{Transcription Editing and Reuse.}
Once a transcript log is generated, perhaps by saving an \file{iESS}
buffer, transcript-mode assists with reuse of part or all of the
entered commands.  It permits editing and re-evaluating the commands
directly from the saved transcript.  This is is useful for
demonstration of techniques as well as for reconstruction of data
analyses.  There currently exist functions within ESS for cleaning
transcripts from S languages back to source code by finding all input
lines and isolating them into an input file.

\paragraph{Help File Editing (R).}
ESS also provides an interface for writing help files for R functions
and packages.  It provides the ability to view and execute embedded R
source code directly from the help file in the same manner as ESS
normally handles code from a source file.  \stexttt{Rd} mode provides
syntax highlighting and the ability to submit code to a running ESS
process, either R or \Splus.


\subsection{ESS Facilitates Common Tasks}
\label{sec:ess-facil-comm}

\subsubsection{Multiple Tools}
\label{sec:multiple-tools}

Statistical software tools are intended for either general data
analysis or for specialized forms of statistical analyses.  The
specialized tools can be orders of magnitude more efficient for
generating data analyses.  This is balanced by the inability of the
specialized tools to perform a wide range of common data analysis
operations.  Tightly coupled inter-operability between these programs
rarely exists, while the need to switch between tools occurs much more
often.  For example, general purpose tools such as R
\citep{ihak:gent:1996} do not perform Bayesian analyses as easily as
tools such as WinBUGS can \citep{SpieThomBest:1999}.  On the other
hand, these specialized tools lack breadth in the range of analyses
and graphics that can be generated.  For this reason, BUGS is
distributed with applications providing loose inter-operability for
simplifying the transfer of output to R or \Splus{} for processing the
results.  Although the interfaces for WinBUGS and R are similar, they
are distinct enough to create tension for the analyst.

Emacs Speaks Statistics (ESS) \citep{ESS} is an extension package for
the Emacs editor which provides a single interface for a variety of
statistical computing tasks.  ESS is optimized for statistical coding
and interactive data analysis.  Statistical coding is the writing of
computer code for data analysis.  This code might be in a compiled
language, such as C or \Fortran, or it might be in an interpreted
language such as \Splus, \SAS, R, \XLispStat, \Perl, or Python.
Entering commands for interactive data analysis is a similar activity.
In either case, text is written in a computer language and sent to a
computer program for evaluation.  The primary difference is that the
results of a small set of commands are of critical interest for review
in the analysis phase, but the results of all commands are of interest
in the coding phase.  Both of these tasks can occur simultaneously,
for example in the use of compiled C code for optimization, which is
called from an interpreted language such as R, where the objective
function to optimize is written.


\subsubsection{Conflicts in keymappings}
\label{sec:confl-keym}

Simple conflicts between interfaces are exemplified by different
keystrokes for editing tasks such as copy and paste, beginning of
line, highlighting regions.  These are sometimes the most aggravating
because our fingers are trained for a primary interface convention
which is continually interfered with.  ESS solves this problem by
providing a uniform keyboard interface.

Complex conflicts between interfaces can involve the coordination of
several data files, multiple statistical software packages, and the
corresponding source code in each of these languages, combined for a
single analysis.  ESS, as part of Emacs, has tools which assist in
this, including support for version and source code control systems,
tools for accessing programs or files on remote machines, and
interfaces to documentation systems including \LaTeX\ and
\textsc{xml}.  In addition, Emacs can assist with, or be programmed to
perform, many tasks related to data cleaning, management, and editing.


\subsubsection{Concurrent Use of Multiple Machines and Operating
  Systems}
\label{sec:conc-use-mult}

It can be useful to have multiple statistical processes running
simultaneously, either on a single machine or a variety of machines.
This capability assists with code design and testing across multiple
versions of statistical software packages, as well as large scale
numerical simulations.  For example, one might want to be connected to
multiple R processes of different versions in order to verify behavior
on different versions of the same software; multiple processes of the
same version to perform test-and-run scenarios where one process is
doing long-term processing while the other is doing short-term
testing; simulations; and to distribute the load over a variety of
remote machines.

\section{History of ESS}
\label{sec:ESS:history}

ESS is a unique example of how open-source products can continue to
develop beyond what their initial authors planned.  There have been 3
generations of developers, and the genealogy is more tree-like than
linear.  There are 2 codebases which formed this project and from
which this descended from; the first is the original Emacs S-mode, and
the second is the original \SAS editing mode.


The ESS codebase was brought to life in 1989 initially as S-mode, to
provide extensions to Emacs for editing S and \Splus{} files.  This
was originally written by Doug Bates, Ed Kademan, Frank Ritter, and
Mike Meyer.  David Smith was the next primary maintainer, and his most
important contribution was to enhance the interface to the \Splus{}
command-line interface and to allow for control of multiple processes
simultaneously.  In 1995, A.J.  Rossini extended S-mode to support
XEmacs, an open-source editor which was derived from Emacs.  At the
same time, extensions from ETH, written by Martin M{\"a}chler, were folded
into what was to become the primary codebase.  By 1996 a uniform
S-mode, available for both Emacs and XEmacs, supported S, \Splus, and
to some extent R.  Kurt Hornik was instrumental in enhancing support
for R, in particular providing Rd mode.

The other ancestor of the ESS codebase was a successful SAS-mode
written by Tom Cook, partially based on ideas, and possibly code, from
a set of Emacs macros written by John Sall of \SAS \textbf{NEED TO GET
  DATES FROM AJR's ARCHIVE!?}.  The initial extension was done by A.J.
Rossini in 1995 to work with XEmacs, and this provided the initial
impetus to provide a uniform codebase for statistical programming.

The extension to a language-independent generic interface was prompted
by the success of R, and the need for an R-mode.  This led to a merger
with the SAS-mode \citep{SASMODE}, and the refactoring of the S-mode
codebase to accommodate multiple languages in a flexible way.

During 1996 and 1997, Richard M. Heiberger further incorporated
SAS-mode into ESS and designed the inferior ESS mode for \SAS.  In
1997, the grand redesign of the internals occurred, where by a generic
means for configuring ESS for new statistical languages was
implemented.

Most of the original work was done for Unix-based statistics packages.
The first example of interprocess communication for Windows using DDE
was constructed in 1998 by Brian Ripley.  Based on this work, Richard
M. Heiberger provided interfaces for Windows versions of \Splus.

In 1998, Rodney Sparapani designed the \SAS batch interaction mode.
Sparapani and Heiberger developed \SAS batch support including function
key behavior that followed SAS Institute's function key definitions,
developed a more comprehensive and efficient syntax highlighting
mechanism using the Emacs Font-lock mode, extended the
\stexttt{*ESS-elsewhere*} functionality to include \SAS processes, and
provided automated error message lookup in the log file.  This provides
a powerful development environment for \SAS.


\section{Discussion}
\label{sec:discussion}

There are two active areas of extensions for user environments.  One
is to enhance the capabilities of the IDE for statistical practice;
this includes implementing such common IDE features as object
browsers, tool-tips, and interfacing cleaning.  The other is to target
appropriate potentially useful programming methodologies for transfer
to statistical practice.

Literate Programming methodologies \citep{Knuth:1992,NRamsey:1994} are
a natural fit for statistical practice.  We refer to the application
to statistical analysis as Literate Statistical Practice
\citep{rossini:dsc:2001}.  The tools used are Noweb
\citep{NRamsey:1994} and either \LaTeX, \textsc{html}, or \textsc{xml}
for documenting and explaining the analysis.  This approach to
programming encourages the use of a literary documentation style to
explain the programming code for the data analysis.  The program can
then be extracted from the documentation text for realizing the
statistical analysis.

Important IDE extensions which should be implemented in future
versions include class browsers, analysis templates, tool-tips, and
similar features.  Class browsers can be thought of as a tree or
outline for presenting datasets, variables and functions in the
context of what they represent; this allows for rapid and appropriate
inspection.  Analysis templates would allow statistics centers and
groups to provide standardized templates for initiating an analysis.
While most IDE features have been developed for object-oriented
languages, the above also can apply to non-object oriented
programming.

ESS is one of the first IDEs intended for statisticians.  It provides
an enhanced, powerful interface for efficient interactive data
analysis and statistical programming.  It is completely customizable
to satisfy individual desires for interface styles as well as being
extensible to additional statistical languages and analysis packages.


\bibliographystyle{plainnat}
%\pdfbookmark[1]{References}{section.7}
\bibliography{essJCGSv2}


\end{document}


%%% Local Variables: 
%%% mode: latex
%%% TeX-master: t
%%% End: 
