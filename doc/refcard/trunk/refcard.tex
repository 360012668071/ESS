%%%- $Header: /scratch/CVS-ARCHIVE/ess-refcard/refcard.tex,v 1.13 2003/10/28 11:54:00 maechler Exp $
\documentclass[twocolumn]{article}
%FIXME\usepackage{multicolumn}
\usepackage{a4}%-Northamerica: {fullpage}
\usepackage{parskip}

\newenvironment{tabI}{\begin{tabular}{p{1.5cm}l}}{\end{tabular}}
\newenvironment{tabTit}[1]{\underline{\bf #1}\\ \begin{tabI}}{\end{tabI}}
\newcommand{\Sect}[1]{\par\noindent\medskip\fbox{\large\textbf{#1}}
  \vskip -.2ex plus 1ex minus 1ex}

\newcommand*{\Ecmd}[1]{$\left\langle \textrm{#1} \right\rangle$}
\newcommand*{\sEcmd}[1]{{\small\Ecmd{#1}}}
\newcommand*{\RET}[0]{\Ecmd{\textsc{ret}}}
\newcommand*{\TAB}[0]{\Ecmd{\textsc{tab}}}

\raggedbottom \pagestyle{empty}

\begin{document}
%FIXME doesn't work: \onecolumn
\begin{center}
  {\LARGE ESS \ \ \ \ {\large
      [\textbf{E}macs \textbf{S}peaks \textbf{Statistics}]}
      \\[.5ex] Reference Card for S and R}

  \smallskip

  {\small updated for ESS 5.x}% {\footnotesize --- needs \em{more} updating!}}
  \\[1ex] {\tiny $ $Date: 2003/10/28 11:54:00 $ $}
           \footnotesize --- as of \today
\end{center}
\begin{enumerate}
\item  \textsc{Nota Bene:} S is the \emph{language},
  R is one \emph{dialect}!
\item This is a list of the more widely used \textbf{key - shortcuts}.
  Many more are available, and most are accessible from the Emacs
  \textbf{Menus} such as \texttt{iESS}, \texttt{ESS}, etc.
\end{enumerate}

%%FIXME starts a new page (when above \onecolumn): \twocolumn
\medskip

\Sect{Interacting with the S process}
%%    ~~~~~~~~~~~~~~~~~~~~~~~~~~~~~~

For use in a process buffer `\texttt{*R*}' ({\tiny inferior-ess-mode}):

\begin{tabI}
  \texttt{\RET   }& Send a command \\
  \texttt{\TAB   }& Complete S object name \\
  \texttt{C-c C-c }& Break \\
  \texttt{C-g}     & interrupt Emacs' waiting for S\\
  \texttt{C-a} / \texttt{C-e} & Beginning / End of command \\
  \texttt{C-c C-u }& Delete this command \\
  \texttt{C-c C-w }& Delete last word
\end{tabI}

\begin{tabTit}{Command history (part of Menu `\texttt{In/Out}')}
  \texttt{M-p   }& Previous            command \\
  \texttt{M-n   }& Next \hspace{1.4em} command \\
  \texttt{$\uparrow =$ \Ecmd{up}}
              & Previous            similar command \\
  \texttt{$\downarrow=$\sEcmd{down}}
              & Next \hspace{1.4em} similar command \\
  \texttt{C-c\RET}& Copy current input \\
%  \texttt{M-C-r }& String search
  \texttt{C-c C-r }& Top of last output \\
  \texttt{C-c C-o }& Delete last output
\end{tabTit}

%% == ESS-transcript mode ?? ---
%\begin{tabTit}{Viewing the transcript}
%  \texttt{M-P   }& Move to last command \\
%  \texttt{M-N   }& Move to next command \\
%  \texttt{C-c C-b }& String search and move \\
%  \texttt{C-c C-v }& Prompt at bottom of screen \\
%\end{tabTit}

\begin{tabTit}{Hot keys}
  \texttt{C-c C-v }& Help for S object \\
  \texttt{C-c C-l }& \textbf{L}oad source file {\small ($+$ error check!)}\\
  \texttt{C-c C-x }& List objects \\
  \texttt{C-c C-s }& Display \texttt{\textbf{s}earch} list \\
  \texttt{C-c C-a }& \textbf{A}ttach a directory \\
  \texttt{C-c C-d }& Edit an object {\small (\textbf{d}ump to file)}
\end{tabTit}

\begin{tabTit}{Others}
  \texttt{C-c `   }& Jump to error after \texttt{C-c C-l}\\
%  \texttt{C-c C-t }& Toggle Tek mode \\
  \texttt{C-c C-q }& Quit from S \\
  \texttt{C-c C-z }& Kill the S process
\end{tabTit}\\[0.5cm]


\Sect{Inside S Transcripts (I + O)}
%%    ~~~~~~~~~~~~~~~~

Inside \texttt{ESS transcript} buffers, (\texttt{*.Rout} files):

\begin{tabI}
  \texttt{\RET} & Send and Move \\
  \texttt{C-c C-n}& Next \hspace{1.4em} prompt \\
  \texttt{C-c C-p}& Previous            prompt \\
  \texttt{C-c C-w}& Clean Region (i.e., transform to input only)
\end{tabI}
\pagebreak
%%%%%%%%%%================================================================


\Sect{Editing S source}
%%    ~~~~~~~~~~~~~~~~
For use in \texttt{ess-mode} edit buffers, (\texttt{*.R} files):

\begin{tabI}
  \texttt{\TAB} & Indent this line \\
  \texttt{C-c\TAB}& Complete S object name \\
  \texttt{M-\TAB} & Complete file- / path- name \\
  \texttt{M-C-a} & Beginning of function \\
  \texttt{M-C-e} & End of function \\
  \texttt{M-C-q} & Indent this expression (use at `\texttt{\{}')\\
  \texttt{M-C-h} & Mark this function
\end{tabI}

\begin{tabTit}{Evaluation commands}
  \texttt{C-c C-l} & Load this buffer -- detect errors !\\
  \texttt{C-c C-n} & \underline{Step through code -- line by line} \\
  \texttt{C-c C-e} & Evaluate an expression \\
  \texttt{C-c C-j} & Evaluate this line \\
  \texttt{C-c M-j} & Evaluate this line and go \\
  \texttt{M-C-x}   & Evaluate this function \\
  \texttt{C-c C-f} & \underline{Evaluate this function} \\
  \texttt{C-c M-f} & Evaluate this function and go \\
  \texttt{C-c C-r} & \underline{Evaluate this region} \\
  \texttt{C-c M-r} & Evaluate this region and go \\
  \texttt{C-c C-b} & Evaluate this buffer \\
  \texttt{C-c M-b} & Evaluate this buffer and go \\
\end{tabTit}

\begin{tabTit}{Others}
  %%-SfS-deleted:  \texttt{M-\TAB} & Complete S object name \\
  \texttt{C-c C-v }& Help for S object \\
  \texttt{C-c C-d }& ``\texttt{dump}'' -- Edit another object \\
  %%\texttt{C-c C-y} & Return to S process \\
  \texttt{C-c C-z} & Return to S process (at prompt)
\end{tabTit}

\begin{tabTit}{At SfS, or activated by \texttt{M-x ess-add-MM-keys}}%-SfS-added
  \texttt{C-c f}   & insert function() definition outline%-SfS-added
\end{tabTit}\\[0.5cm]                                   %-SfS-added

\Sect{Reading help files}
%%    ~~~~~~~~~~~~~~~~~~

For use in `\texttt{*help[R]($\ldots$)*}' help buffers:

\begin{tabI}
  \texttt{SPC} & Next page \\
  \texttt{DEL} & Previous page \\
  \texttt{b}   & Previous page (`\textbf{b}ack')\\
  \texttt{/}   & Search forwards \\
  \texttt{n}   & \textbf{N}ext section \\
  \texttt{p}   & \textbf{P}revious section \\
  \texttt{s}   & \textbf{S}kip (`jump') to a named section \\
  \texttt{s e} & e.g., \textbf{s}kip to ``\texttt{\textbf{E}xamples:}'' \\
  \texttt{l}   & \underline{Evaluate one `Example' \textbf{l}ine} \\
  \texttt{r}   & Evaluate current \textbf{r}egion \\
  \texttt{h}   &\textbf{H}elp on another object \\
  \texttt{?}   & Help for this mode \\
  \texttt{q}   & Return to S process (`\textbf{q}uit) \\
  \texttt{x}   & Kill this buffer and return (`e\textbf{x}it) \\
\end{tabI}

\end{document}
